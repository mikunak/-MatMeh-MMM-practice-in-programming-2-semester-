“Двунаправленная Очередь”
a)Как устроены элементы
дэк состоит из множества маленьких массивов, которые динамически расположены в ячейках памяти, то есть не лежат подряд.
Если мы хотим добавить элемент, а массив занят, то создается еще один массив, и меняется указатель с NULL на этот массив.
Можно переходить от одного массива к другому, так как есть указатели на left и right.
Также можно добавлять и извлекать элементы с разных сторон(back и front), так как мы запоминаем эти указатели. 
За счет того, что используются массивы, доступ к элементам происходит очень быстро. 

b)Как устроены push/pop
для front и back
push - добавление элемента, можно добавлять с любой стороны. 
У элемента, который мы добавляем, есть адрес.
Мы меняем адрес предыдущего элемента на адрес этого элемента. Таким образом мы добавили элемент в дэк.
pop - извлечение, можно извлекать с любой стороны. 
У предыдущего элемента меняется указатель на NULL, извлекаем нужный элемент.
Если добавлять элемент в какое-то определенное место, то у элемента, который расположен слева, меняем указатель на добавленный. У добавленного - на правый.
Если извелкать -//-, то указатель слева будет показывать на next->next вместо next.Записываем элемент в переменную, удаляем ячейку, возвращаем результат.

c)Какие оценки сложности работы СД. 
методы push и pop занимают O(1) времени.